\documentclass[a4paper,10pt]{article}
\usepackage[utf8]{inputenc}
\usepackage{fullpage}
\usepackage{amsmath}

%opening
\title{MoData - A Distributed Hash Table Based Filestore for Everyone}
\author{Merritt Boyd, Russell Cohen, Joseph Lynch}

\begin{document}

\maketitle

\section{Problem}
Persistent, reliable, online storage solutions such as Dropbox, Box, and Google 
Drive are all for-pay services controlled by large multinational companies.  
Many users do not wish to pay for these services and do not want their data 
controlled by large companies.  At the same time, most computer users have 
excess disk storage and network bandwidth. It seems that a system where people 
share their bandwidth or storage in exchange for a free, peer-hosted cloud 
storage alternative is very viable. We strive to create a system to allow 
access to your files wherever you go, without relying on a central party 
required to host files.

The storage solution should prevent the users files from being compromised 
either by malicious corruption and make best possible efforts to prevent loss.

\section{Solution}
We present a decentralized peer-to-peer key-value storage system targeted 
towards nodes with low availability. The software is intended to be run on 
personal computers, especially laptops, which may frequently disconnected from 
the network and possibly powered off.  By heavily replicating user data 
across the cluster, we aim to maintain availability of user data from any device 
even in the absence of much of the peak size of the network.

Our system stores files indexed by a key which is simply a hash of the file’s 
contents.  As we are not optimizing for peer-to-peer file sharing, we provide no 
in-system mechanism for listing a user’s files -- this metadata must be 
maintained and transferred by the user out-of-band, for instance on a USB flash 
drive.  This provides extra security by requiring a token not stored in the 
cloud to access files.

\section{Implementation}
The project is split into two main components, the server and the client.  The 
server is a straightforward RESTful API that implements a Kademlia based DHT, 
and the client is a collection of python scripts that provide an easy to use 
file interface as well as additional redundancy.

\subsection{Server - Kademlia DHT}
The server implements the standard Kademlia Distributed Hash Table algorithm 
\cite{kademlia}.  Kademlia can automatically handle nodes joining, leaving and 
publishing key/value pairs to the network.  The algorithm is fundamentally peer 
to peer, and nodes join by bootstrapping in on old nodes.  Each node is given 
an identifier, and then consistent hashing based on the XOR metric is used to 
evenly distribute keys among available nodes.  All keys and node-ids are 160 
bit identifiers. The ``closeness''  metric of XOR is used to define that a key 
K's closeness to a node N is defined as $K \oplus N$.  Because XOR satisfies the 
triangle identity, this guarantees that there is a globally correct set of 
closest nodes for any given key.

All distributed operations must first do a distributed find node, which 
collects nodes nearby to the key by calling primitive find-node on hosts that a 
particular peer knows about.  By chaining these requests together, we can build 
a list of nearest nodes in O(log(n)) time, where n is the size of the network.  
After finding the list of nearest nodes, stores then send primitive store 
operations to those nodes, and find-values send primitive find-value calls.  
All changes are persisted to disk.

Our server exposes the following REST API to meet the Kademlia specification.  
\begin{center}
\begin{tabular}{| l | c | p{5cm} |}
\hline                       
\textbf{Endpoint} & \textbf{Type}& \textbf{Action}\\
\hline
/find-node/\{node-id\} & GET & Returns locally known nodes near the node-id\\
\hline
/find-value/\{key\} & GET & Returns the locally known value for a key\\
\hline
/store?key=\{key\}\&data=\{data\} & POST & Locally stores a value for a key\\
\hline
/ping & GET & Health check, also updates contacts\\
\hline
/distributed/find-node/\{node-id\} & GET & Returns globally known nodes near 
the node-id\\
\hline
/distributed/find-value/\{key\} & GET & Returns the globally known value for a 
key\\
\hline
/distributed/store?key=\{key\}\&data=\{data\} & POST & Stores the value for the
key on nearby nodes 

\end{tabular}
\end{center}

\subsection{Client - Erasure Coding File Transfer}
To help cope with low availability, files are erasure encoded. This gives us a 
knob which we can turn to set the percentage of chunks that the system can 
loose, but still recover the file. Our client uses using Reed-Solomon codes, an 
optimal erasure encoding scheme. It produces a number of chunks. From those 
chunks it creates a metadata file containing the hash for each chunk. A hash is 
enough information to retrieve the data from the DHT. The client then 
attempts to get sufficient chunks to decode the file, decoding the resulting 
chunks client-side.

\section{Challenges and Analysis}
The most challenging part of this project was getting a functional kademlia 
based DHT and building additional availability guarantees into the system.

Because we are assuming that the target consumer will be using laptops, we 
chose usage analysis of wifi networks as the starting point for availability 
analysis. In particular, we assume that most wifi users behave in similar 
fashions to that of the Mountain View wide wifi provided by Google \cite{wifi}.  
Using these assumptions, we can do a quick back of the envelope calculation for 
data loss. Assuming that we republish keys every hour, we only have to care 
about losing data in a given hourly period, since at the end of that period we 
will re-replicate the key.  We can choose the replication factor of Kademlia to 
trade off bandwidth with availability, and a commonly chosen value is 20.  Due 
to consistent hashing, we can say that node failures are independent, and that 
the probability of a given node failure is equal to the probability of failure 
conditioned on the node type.  Node type is split into long lived, which are on 
all day, medium lived which are on for between 100 and 1000 minutes, and short 
lived which are on for less than an hour.  
\begin{multline}
Pr(lose\_key) = Pr(k\_nodes\_fail) = Pr(single\_failure)^{20}
\\
Pr(single\_failure) =
Pr(failure | long\_lived)*Pr(long\_lived) +
\\
Pr(failure | medium\_lived)*Pr(medium\_lived) +
\\
Pr(failure | short\_lived)*Pr(short\_lived)
\\
\end{multline}

We can estimate the duration probabilities from the Wifi paper Figure 8 
\cite{wifi} and to be as conservative as possible we say that $Pr(failure)=1$ 
for all types except long lived, for which $Pr(failure)=0$, this yields 
$Pr(single\_failure)= 0*.2 + 1*.6 + 1*.2 = .8$, and 
therefore $Pr(lose\_key) \approx 0.01$.  Due to erasure coding with code rate 
of 1/2, the probability of losing a file with N chunks is 
$Pr(lose\_key)^{\frac{N}{2}}$. 
This means that to lose a 1 Megabyte file, one must lose 125 keys (assuming a 
4kb chunk size). Losing access to a file therefore has probability 
$Pr(data\_loss) \approx 5*10^{-243} \approx 0$.  Even if access is lost due to 
massive correlated failure, it will be restored once nodes return to the 
network.

\pagebreak
\section{Bibliography}
\bibliographystyle{IEEEtran}
\bibliography{writeup}

\end{document}
